\documentclass{article}
\usepackage[utf8]{inputenc}
\usepackage{amsmath}
\usepackage{amssymb}
\usepackage{amsthm}
\usepackage{mathrsfs}
\usepackage{tipa}
\usepackage{graphicx}
\usepackage{stix}


\title{Rubik's Cube DeCal Take-Home Exam Solutions}
\author{April 14, 2021}
\date{}

\begin{document}
\maketitle

Note: cycle composition is done from left to right. So to compose cycles $a$ and $b$, $ab$ represents first doing $a$, then doing $b$. 
\\\\
\textbf{\textit{Question 1}}
Your tutor Manu tells his friend Zod about a new group which he came up with. He says his group is $(\mathbb{Z} \setminus 0, \cdot).$ The group elements are the integers except zero, and the group operation is multiplication. Zod says Manu is wrong, and his group is not a valid group. Who is right and why? Select each item which is completely correct.
\begin{itemize}
    \item Manu is right since the group is a valid group.
    \item Zod is right. The group is not valid because it is not associative (A group $G$ is associative if for any elements $f, g,$ and $h$ in $G, (f * g) * h = f * (g * h)$).
    \item Zod is right. The group is not valid because there is not an identity element (A group $G$ has an identity if there exist an identity element $e$ in $G$ such that for all $g$ in $G, g * e = e * g = g$).
    \item Zod is right. The group is not valid because not all elements have an inverse (All elements in a group $G$ have an inverse if for all elements $g$ in $G$, there exists an element $g^{-1}$ in $G$ such that $g * g^{-1} = g^{-1} * g = e).$
    \item Zod is right. The group is not valid because because the group is not closed (A group $G$ is closed if for any elements $f, g$ in $G$, the element $f * g$ is also in $G$). 
    \item Zod is right. The group is not valid because because the group is not finite (A group $G$ is finite if it has a finite number of elements). 
\end{itemize}
%\textbf{Answer:} Only the fourth one is right. Multiplication on integers is associative, 1 is an identity element, and the group is closed. However, not all elements have an inverse since that would require fractions (the inverse of 5 is 1/5). Being finite is not required of a group.
\\\\
\textbf{\textit{Question 2}}
The commutator $[f, g]$ on group elements $f, g$ is defined as $$[f, g] := fgf^{-1}g^{-1}.$$ True or False: For an Abelian group, the commutator is equivalent to the identity, regardless of which elements we input to it.
%\textbf{Answer:} True. Since the group is Abelian, we can commute the elements and cancel the elements.
\\\\
\textbf{\textit{Question 3}} Let $S_6$ be the symmetric group on 6 elements. Recall it is the group of all permutations on 6 elements. Consider its subgroup $\langle (1\ 2\ 3), (4\ 5) \rangle$ generated by the 3-cycle $(1\ 2\ 3)$ and the 2-cycle $(4\ 5)$. How many right cosets does this subgroup have in $S_6$? \\
a. 6 \quad b. 120 \quad c. 240 \quad d. 720
%\textbf{Answer:} 120. The subgroup has order LCM(2, 3) = 6, and by Lagrange's theorem the number of cosets is equal to 6! / 6 = 120.
\\\\
\textbf{\textit{Question 4}} Let $\mathcal{R}$ be the group of Rubik's cube permutations. Consider its 2-gen subgroup, $\langle R, U \rangle$, generated by $R$ and $U$ moves. Let $n$ be the order of this subgroup. Does 30 divide $n$?
(Hint: consider the order of the moves $RUR'U'$ and $RU'R'U'$.) \\
%\textbf{Answer:} Yes. Since there is a subgroup of order 6 (R U R' U') and order 5 (R U' R' U), the order of the subgroup $\langle R, U \rangle$ must be divisible by 5 and divisible by 6. This implies it is also divisible by the product of the two, and $5 \times 6 = 30.$
\\\\
\textbf{\textit{Question 5}}
Let
\[
\sigma=
\begin{pmatrix}
1 & 2 & 3 & 4 & 5 \\
2 & 3 & 1 & 5 & 4
\end{pmatrix}
\]
Which of the following statements about $\sigma$ are correct?
\begin{itemize}
\item Multiplying $(1\;2), (1\;3),$ and $(4\;5)$ in any order produces $\sigma$.
\item $\sigma$ has order 3.
\item $\sigma = (1\;2)(4\;5)(1\;3)$.
\item The longest cycle in the cycle decomposition of $\sigma$ has odd parity.
\end{itemize}
%Answer: Option 3 only. Transpositions do not generally commute so option 1 is incorrect. $\sigma$ can be decomposed into $(1\;2\;3)(4\;5)$ so it has order $2 \cdot 3 = 6$, not 3. Transpositions of disjoint cycles can be interleaved so $\sigma=(1\;2)(1\;3)(4\;5)=(1\;2)(4\;5)(1\;3)$. The longest cycle in the decomposition is $(1\;3\;2)$ which has even parity.
\\\\
\textbf{\textit{Question 6}}
Consider the following algorithm
\begin{verbatim}
R U' R U R U R U' R' U' R2
\end{verbatim}
What are the edge/corner parity after the algorithm? 
\begin{itemize}
    \item odd edges, odd corners
    \item even edges, odd corners
    \item odd edges, even corners
    \item even edges, even corners
\end{itemize}
%\textbf{Answer:} Even edges, even corners. We know the parity of the edges/corners must be the same, and we observe the corners have not been changed from the solved state. Hence, the parity of corners is even (it can be given as 0 transpositions) which means the parity of edges is also even. 
\\\\
\textbf{\textit{Question 7}}
Define 
\[ \sigma = \begin{pmatrix}
 1 & 2 & 3 & 4 & 5 & 6 \\
 2 & 4 & 3 & 1 & 6 & 5
\end{pmatrix}\]
Express $\sigma^2 = \sigma \cdot \sigma$ in cycle notation. 
\begin{itemize}
    \item $(14562)$
    \item $(124)(56)$
    \item $(421)$
    \item $(421)(56)$
\end{itemize}
%\textbf{Answer:} (421). We see that doing the original cycle is a transposition of 5 and 6 along with a 3-cycle between 1, 2 and 4. We can write this 3 cycle as (1 2 4). When we do it twice, we see it is the inverse (4 2 1).
\\\\
\textbf{\textit{Question 8}}
\question Define 
\[ \sigma = \begin{pmatrix}
 1 & 2 & 3 & 4 & 5 & 6 \\
 2 & 4 & 3 & 1 & 6 & 5
\end{pmatrix}\]
What is the order of $\sigma$? \\
\begin{itemize}
    \item $1$
    \item $2$
    \item $3$
    \item $4$
    \item $6$
\end{itemize}
%\textbf{Answer:} 6. We have an independent 3 cycle and 2 cycle so the order is 6.
\\\\
\textbf{\textit{Question 9}} What's the order of the subgroup generated by the following algorithm?
\begin{verbatim}
    R U R' U' R' F R2 U' R' U' R U R' F'
\end{verbatim}
\begin{itemize}
    \item $1$
    \item $2$
    \item $3$
    \item $4$
\end{itemize}
%\textbf{Answer:} 2. Doing the algorithm twice leads to a solved state. Alternatively, the algorithm executes two independent transpositions so it is its own inverse.
\\\\
\textbf{\textit{Question 10}}
\question Find the inverse of the following algorithm
\begin{verbatim}
R U' R' U R U' R'
\end{verbatim}

\begin{itemize}
    \item \texttt{R U R' U' R U R'}
    \item \texttt{R U R U R U R}
    \item \texttt{R U' R' U' R' U' R'}
    \item \texttt{R' U' R' U R U' R'}
\end{itemize}
%\textbf{Answer:} R U R' U' R U R' is the inverse. To find this, go through the algorithm in reverse and invert each move.
\\\\
\textbf{\textit{Question 11}}
\question Let $|G|$ denote the order of the Rubik's cube group. What's the biggest prime that divides $|G|$? \\
\begin{itemize}
    \item $11$
    \item $7$
    \item $2$
    \item $5$
\end{itemize}
%\textbf{Answer:} 11. See the reading for the order of the group. The largest prime in the factorization is 11.
\\\\
\textbf{\textit{Question 12}}
which of the following two sets of moves result in the same state?
\begin{itemize}
    \item (M2 E2) and (R2 L2 U2 B2)
    \item (R U R' U')(R U R' U')(R U R' U')(R U R' U')(R U R' U') and (U R U' R')
    \item F (R U R' U')F' and F' (L' U' L U) F
    \item R U R' U R U2 R' and R' U' R U' R' U2 R
\end{itemize}
\end{document}