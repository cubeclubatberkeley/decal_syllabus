\documentclass{article}
\usepackage[utf8]{inputenc}
\usepackage{amsmath}
\usepackage{amssymb}
\usepackage{amsthm}
\usepackage{mathrsfs}
\usepackage{tipa}
\usepackage{graphicx}
\usepackage{stix}


\title{Rubik's Cube DeCal Take-Home Exam Solutions}
\author{April 14, 2021}
\date{}

\begin{document}
\maketitle

\begin{proof}[\textbf{Question 1}]
\textbf{A group is a set of objects with a binary operator which has satisfy various conditions. Suppose $G$ is a set of elements with a binary operation $\star$. Select each condition which is required for $(G, \star)$ to be an group.}
    \begin{itemize}
    \item For any elements $f, g,$ and $h$ in $G, (f * g) * h = f * (g * h)$.
    \item There exist an identity element $e$ in $G$ such that for all $g$ in $G, g * e = e * g = g$.
    \item For all elements $g$ in $G$, there exists an element $g^{-1}$ in $G$ such that $g * g^{-1} = g^{-1} * g = e.$
    \item For any elements $f, g$ in $G, f * g = g * f$.
    \item For any elements $f, g$ in $G$, the element $f * g$ is also in $G$. 
    \end{itemize}
\textbf{Answer:} All but the fourth one. The fourth one would be true for an Abelian group, but not groups in general.
\end{proof}

\begin{proof}[\textbf{Question 2}]
\textbf{What is the order of the following cycle given in cycle notation (Please enter an integer):
\[\begin{pmatrix} 1 & 2 & 3 & 4 & 5 \end{pmatrix}\begin{pmatrix} 6 & 7 & 8 \end{pmatrix}\]}
\textbf{Answer:} 15. The first cycle has order 5 and the second has order 3, so the order is the LCM (least common multiple) of the two which is 15.
\end{proof}

\begin{proof}[\textbf{Question 3}]
\textbf{True or False: The following cycle has an even parity:
\[\begin{pmatrix} 1 & 2 & 3 & 4 \end{pmatrix}\]}
\textbf{Answer:} False. Writing the cycle as transpositions can be done as \begin{pmatrix} 1 & 4 \end{pmatrix} \begin{pmatrix} 1 & 3 \end{pmatrix} \begin{pmatrix} 1 & 2 \end{pmatrix}. There are three transpositions here so the parity is odd. 
\end{proof}

\newpage

\begin{proof}[\textbf{Question 4}]
\textbf{True or False: The cube always has an even parity (the number of cubies exchanged from the starting position is always even).}\\
\textbf{Answer:} True. See the reading on parity. 
\end{proof}

\begin{proof}[\textbf{Question 5}]
\textbf{Find the order of the subgroup of the Rubik's cube generated by the following element:\begin{center} (R' L F R L' U') \end{center}}\\
\textbf{Answer:} 5. Just repeat the sequence 5 times. Alternatively, doing the sequence once allows us to observe this is a 5-cycle on edges. 
\end{proof}

\begin{proof}[\textbf{Question 6}]
\textbf{Which of the following choices is the inverse of the following moves: \begin{center} (R U L R') \end{center}}
\begin{itemize}
    \item R' U' L' R 
    \item R U L R 
    \item R L' U' R' 
    \item R' L' U' R' 
\end{itemize}
\textbf{Answer:} The third one. To invert, go backwards through the sequence and invert each move. 
\end{proof}

\begin{proof}[\textbf{Question 7}]
\textbf{True or False: The following two group elements are the same:}
\begin{enumerate}
    \item U D U
    \item U2 D
\end{enumerate}
\textbf{Answer:} True. A group element of the Rubik's Cube group is given as a state of the cube. Because turns on opposing faces commute with one another, these two elements are the same state.
\end{proof}

\begin{proof}[\textbf{Question 8}]
\textbf{Which of these choices is a valid cycle decomposition of the following permutation?}
\[\begin{pmatrix} 1 & 2 & 3 & 4 & 5 \\
3 & 2 & 1 & 5 & 4\end{pmatrix}\]
\begin{itemize}
    \item (1 3) (4 5) 
    \item (1 2 3) (4 5) 
    \item (1 3 5 4) 
    \item (3 2 5) 
\end{itemize}
\textbf{Answer:} The first one. The permutation is the same as two transpositions.
\end{proof}

\begin{proof}[\textbf{Question 9}]
\textbf{What is the corner/edge parity of the cube after applying the following move to the solved cube?\\
R}
\begin{itemize}
    \item Corners are even, edges are even.
    \item Corners are odd, edges are odd. 
    \item Corners are even, edges are odd. 
    \item Corners are odd, edges are even. 
\end{itemize}
\textbf{Answer:} Corners are odd, edges are odd. This produces a 4 cycle on edges and corners, and a 4 cycle has an odd parity (see Q3).
\end{proof}
\end{document}