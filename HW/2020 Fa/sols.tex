\documentclass{article}
\usepackage[utf8]{inputenc}
\usepackage{amsmath}
\usepackage{amssymb}
\usepackage{amsthm}
\usepackage{mathrsfs}
\usepackage{tipa}
\usepackage{graphicx}
\usepackage{stix}


\title{Rubik's Cube DeCal Homework Solutions}
\author{November 19, 2020}
\date{}

\begin{document}
\maketitle

\begin{proof}[\textbf{Homework 1}]
\textbf{1. Invert the following sequence of moves (Try it on your cube or on alg.cubing.net, you should see the original + the inverse does nothing and vice versa).
\begin{center} R U R' U R U2 R' \end{center}}

\noindent \textbf{2. Show that the Rubik's cube group is non-Abelian (non-commutative). (Hint: Find 2 sequences of moves A and B so that AB does not result in the same permutation as BA.)}

\textbf{Answer 1:} The inverse is R U2 R' U' R U' R'

\textbf{Answer 2:} R U does not give the same permutation as U R.
\end{proof}

\begin{proof}[\textbf{Homework 2}]
\textbf{Find a cycle decomposition of the following permutation (any cycle decomposition will do):
\[\begin{pmatrix} 1 & 2 & 3 & 4 & 5 \\
3 & 5 & 4 & 1 & 2\end{pmatrix}\]}

\
\textbf{Answer:} There are multiple correct answers since you can write the cycles in either order, and (2 5) is the same as (5 2).

\\
\noindent The correct answers are (2 5) (1 3 4) or (5 2) (1 3 4) or (1 3 4) (2 5) or (1 3 4) (5 2).
\end{proof}

\begin{proof}[\textbf{Homework 3}]
\textbf{Apply the following moves to your cube:
\begin{center} R U R' U' \end{center}}

\noindent\textbf{(If your cube isn't solved you can type the algorithm into alg.cubing.net to see what permutation it does.)}

\noindent\textbf{What is the parity of the corners in the permutation that this does to the cube? What is the parity of the edges? (They should be both even or both odd!)}

\noindent\textbf{Describe the cycles of this permutation.}

\textbf{Answer:} The sequence of moves does a 3-cycle of edges, and two 2-cycles of corners. An odd cycle has even parity, so the edges have even parity. A single 2-cycle has odd parity, so 2 2-cyles have odd + odd = even parity. Thus the parity of the edges and corners are both even.
\end{proof}

\newpage

\begin{proof}[\textbf{Homework 4}]
\textbf{1. Find the order of the subgroup generated by the following element:
\begin{center} (F R U R' U' F') \end{center}}

\noindent\textbf{2. Give a distinct subgroup with the same order as the subgroup given in 1.}

\textbf{Answer 1:} Starting from a solved cube and repeating the sequence 6 times gets the cube back to solved, so the order is 6. Alternatively, the algorithm does a 3-cycle of edges and two 2-cycles of corners so the order is the least common multiple of these cycle numbers, which is $\text{lcm}(2,2,3) = 2 \cdot 3 = 6$.

\textbf{Answer 2:} There are numerous correct answers to this question, many of which can be obtained through what is called "conjugation". You can also mirror the algorithm or execute it on a different face.

\noindent Conjugation examples: (R U R' U') or (R U2 R' F')(F R U R' U' F')(F R U2 R')

\noindent Mirrored: (F' L' U' L U F)

\noindent Executed on the right face: (F D R D' R' F')

\noindent Some unrelated solutions: (R2 U2) or (F2 D R2 U' R2 F2 D' L2 U L2)
\end{proof}
\end{document}